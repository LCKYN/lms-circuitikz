\documentclass{standalone}
\usepackage[usenames]{color} %used for font color
\usepackage[rgb]{xcolor}
\usepackage{amssymb} %maths
\usepackage{amsmath} %maths
\usepackage[utf8]{inputenc} %useful to type directly diacritic characters
\usepackage{tikz}
\usepackage{circuitikz}
\usetikzlibrary{arrows}

\makeatletter
\ctikzset{current arrow color/.initial=black}% create key

\let\old@circ@drawcurrent=\pgf@circ@drawcurrent
\def\pgf@circ@drawcurrent{\old@circ@drawcurrent}

\pgfdeclareshape{currarrow}{
\anchor{center}{
\pgfpointorigin
}
\anchor{tip}{
\pgfpointorigin
    \pgf@circ@res@step = \pgf@circ@Rlen
        \divide \pgf@circ@res@step by 16
\pgf@x  =\pgf@circ@res@step
}
\behindforegroundpath{      

\pgfscope
    \pgf@circ@res@step = \pgf@circ@Rlen
    \divide \pgf@circ@res@step by 16

    \pgfpathmoveto{\pgfpoint{-.7\pgf@circ@res@step}{0pt}}
    \pgfpathlineto{\pgfpoint{-.7\pgf@circ@res@step}{-.8\pgf@circ@res@step}}
    \pgfpathlineto{\pgfpoint{1\pgf@circ@res@step}{0pt}}
    \pgfpathlineto{\pgfpoint{-.7\pgf@circ@res@step}{.8\pgf@circ@res@step}}
    \pgfpathlineto{\pgfpoint{-.7\pgf@circ@res@step}{0pt}}           
    \pgfsetcolor{\pgfkeysvalueof{/tikz/circuitikz/current arrow color}}
    \pgfusepath{draw,fill}

\endpgfscope
}
}
\pgfdeclareshape{flowarrow}{
    \anchor{center}{\pgfpointorigin}
    \anchor{tip}{
    \pgfpointorigin
        \pgf@circ@res@step = \pgf@circ@Rlen
            \divide \pgf@circ@res@step by 16
    \pgf@x  =\pgf@circ@res@step
    }
\behindforegroundpath{
    \pgfscope
        \pgf@circ@res@step = \pgf@circ@Rlen
        \divide \pgf@circ@res@step by 4
        \pgfpathmoveto{\pgfpoint{-\pgf@circ@res@step}{0pt}}
        \pgfpathlineto{\pgfpoint{\pgf@circ@res@step}{0pt}}
        %\pgfsetcolor{\pgfkeysvalueof{/tikz/circuitikz/color}}
  \pgfsetcolor{\pgfkeysvalueof{/tikz/circuitikz/current arrow color}}
        \pgfusepath{draw}
        \pgftransformshift{\pgfpoint{\pgf@circ@res@step}{0pt}}
        \pgfnode{currarrow}{tip}{}{}{\pgfusepath{fill}}
    \endpgfscope
}
}
\makeatother

\tikzset{voltage dir=RP}
\tikzstyle{every node}=[font=\Large]

\ctikzset{label/align = straight}
\ctikzset{current arrow color/.initial=black}

% \newcommand*{\xMin}{-1}%
% \newcommand*{\xMax}{7}%
% \newcommand*{\yMin}{-1}%
% \newcommand*{\yMax}{7}%

\begin{document}
\begin{circuitikz}[american voltages]
    % grid
    % \foreach \i in {\xMin,...,\xMax} {
    %     \draw [very thin,gray] (\i,\yMin) -- (\i,\yMax)  node [below] at (\i,\yMin) {$\i$};
    % }
    % \foreach \i in {\yMin,...,\yMax} {
    %     \draw [very thin,gray] (\xMin,\i) -- (\xMax,\i) node [left] at (\xMin,\i) {$\i$};
    % }

    % x
    \draw (0,0) to[american current source, l=$I_{S1}$] ++(0,5) ;
    \draw (2,0) to[R, l=$R_3$] ++(0,2.5) to[R, l=$R_2$] ++(0,2.5) ;
    \draw (5,0) to[R, l=$R_4$] ++(0,5) ;
    \draw (7,0) to[american current source, l=$I_{S2}$] ++(0,5) ;

    % y
    \draw (0,0) to[short,-*] ++(2,0) to[short,-*] ++(1.5,0) node[ground]{} to[short,-*] ++(1.5,0) to[short,-] ++(2,0);
    \draw (0,5) to[short] ++(2,0) to[R,l=$R_1$] ++(3,0) to[short] ++(2,0);
    
    % node
    % \draw[circuitikz/current arrow color=purple](14,0) to[R, l=$R_4$, f_<=$\color{purple}{i_4}$] ++(0,5);
    \draw(2,5) node[circ, color=purple]{} node[anchor=south west,color=purple] {$v_1$};
    \draw(5,5) node[circ, color=purple]{} node[anchor=south west,color=purple] {$v_2$};

\end{circuitikz}

% \begin{circuitikz}
%     \draw(0,0) node[circ, color=purple]{} node[anchor=south west,color=purple] {2};
% \end{circuitikz}

\end{document}