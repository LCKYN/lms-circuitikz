\documentclass{standalone}
\usepackage[usenames]{color} %used for font color
\usepackage[rgb]{xcolor}
\usepackage{amssymb} %maths
\usepackage{amsmath} %maths
\usepackage[utf8]{inputenc} %useful to type directly diacritic characters
\usepackage{tikz}
\usepackage{circuitikz}
\usetikzlibrary{arrows}

\makeatletter
\ctikzset{current arrow color/.initial=black}% create key

\let\old@circ@drawcurrent=\pgf@circ@drawcurrent
\def\pgf@circ@drawcurrent{\old@circ@drawcurrent}

\pgfdeclareshape{currarrow}{
\anchor{center}{
\pgfpointorigin
}
\anchor{tip}{
\pgfpointorigin
    \pgf@circ@res@step = \pgf@circ@Rlen
        \divide \pgf@circ@res@step by 16
\pgf@x  =\pgf@circ@res@step
}
\behindforegroundpath{      

\pgfscope
    \pgf@circ@res@step = \pgf@circ@Rlen
    \divide \pgf@circ@res@step by 16

    \pgfpathmoveto{\pgfpoint{-.7\pgf@circ@res@step}{0pt}}
    \pgfpathlineto{\pgfpoint{-.7\pgf@circ@res@step}{-.8\pgf@circ@res@step}}
    \pgfpathlineto{\pgfpoint{1\pgf@circ@res@step}{0pt}}
    \pgfpathlineto{\pgfpoint{-.7\pgf@circ@res@step}{.8\pgf@circ@res@step}}
    \pgfpathlineto{\pgfpoint{-.7\pgf@circ@res@step}{0pt}}           
    \pgfsetcolor{\pgfkeysvalueof{/tikz/circuitikz/current arrow color}}
    \pgfusepath{draw,fill}

\endpgfscope
}
}
\pgfdeclareshape{flowarrow}{
    \anchor{center}{\pgfpointorigin}
    \anchor{tip}{
    \pgfpointorigin
        \pgf@circ@res@step = \pgf@circ@Rlen
            \divide \pgf@circ@res@step by 16
    \pgf@x  =\pgf@circ@res@step
    }
\behindforegroundpath{
    \pgfscope
        \pgf@circ@res@step = \pgf@circ@Rlen
        \divide \pgf@circ@res@step by 4
        \pgfpathmoveto{\pgfpoint{-\pgf@circ@res@step}{0pt}}
        \pgfpathlineto{\pgfpoint{\pgf@circ@res@step}{0pt}}
        %\pgfsetcolor{\pgfkeysvalueof{/tikz/circuitikz/color}}
  \pgfsetcolor{\pgfkeysvalueof{/tikz/circuitikz/current arrow color}}
        \pgfusepath{draw}
        \pgftransformshift{\pgfpoint{\pgf@circ@res@step}{0pt}}
        \pgfnode{currarrow}{tip}{}{}{\pgfusepath{fill}}
    \endpgfscope
}
}
\makeatother

\tikzset{voltage dir=RP}
\tikzstyle{every node}=[font=\Large]

\ctikzset{label/align = straight}

%name
\newcommand*{\vsname}[1]{$V_{S#1}$}
\newcommand*{\isname}[1]{$I_{S#1}$}
\newcommand*{\rname}[1]{$R_{#1}$}
\newcommand*{\lname}[1]{$L_{#1}$}
\newcommand*{\cname}[1]{$C_{#1}$}
% value
\newcommand*{\rv}[1][]{$R_{#1}$}
\newcommand*{\lv}[1][]{$L_{#1}$}
\newcommand*{\cv}[1][]{$C_{#1}$}

\newcommand{\nodev}[3]{
    \draw(#1,#2) node[circ, color=magenta]{} node[anchor=south west,color=magenta] {$v_{#3}$};
}

\newcommand{\nodevv}[3]{
    \draw(#1,#2) node[circ, color=magenta]{} node[anchor=south east,color=magenta] {$v_{#3}$};
}

\newcommand{\nodevs}[3]{
    \draw(#1,#2) node[circ, color=magenta]{} node[anchor=north,color=magenta] {$v_{#3}$};
}

\newcommand{\loopc}[3]{
    \draw[thin, <-, >=triangle 45,color=purple] (#1,#2)node{$i_{#3}$}  ++(-70:0.6) arc (-70:160:0.6);
}